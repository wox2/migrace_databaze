%%This is a very basic article template.
%%There is just one section and two subsections.
\documentclass{article}
\usepackage[czech, english]{babel}
\usepackage[T1]{fontenc} % pouzije EC fonty \usepackage[utf8]{inputenc}
\usepackage{graphicx} 
\usepackage[utf8]{inputenc}
\usepackage{amsmath}  %package potrebny pro matiku
\usepackage{amsfonts}  %package obsahujici symboly mnozin 
\usepackage{hyperref}
\begin{document}

\section{Vypracoval}
Martin Lukeš \\
aka woxie\\
7.10. 2012


\section{Úlohy - původ}
Vypracované úlohy 2.5, 2.6, 2.10 pochází z Boydova kurzu - \\ viz
\hyperref[Boydova
kurzu]{http://www.stanford.edu/class/ee263/hw/263homework.pdf}.

\section{Úloha 2.19}
Máme dvě matice A a B, přičemž B je inverzní k A. Všechny prvky obou matic jsou
nezáporné.
\subsection{Předpoklad}
Z prvního pohledu to vypadá, že jednotková matice je řešením úlohy, tudíž
předpokládejme, že matice A=B, pro kterou platí : $a_{ii} = 1,\ a_{ij} = 0
\forall\ i \neq j ,\ i \in \mathbb{R},\ j \in \mathbb{R} $
\subsection{Pozorování} Výpočet inverzní matice B k regulární matici A. Pro
matici A platí :
$a_{ij} > 0 \forall\ i \neq j ,\ i \in \mathbb{R},\ j \in \mathbb{R} $ . K
výpočtu inverze k A je používána Gauss/Jordanova eliminační metoda, kdy za sebe
napíšeme eliminační metoda.

\subsection{Řešení}


\subsection{Another subtitle}

More plain text.


\end{document}
