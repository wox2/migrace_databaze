\documentclass[a4paper,12pt]{report}
\usepackage[czech, english]{babel}
\usepackage[utf8]{inputenc}
\usepackage{mnsymbol}
\usepackage{fullpage}
\begin{document}
\begin{flushright}
\begin{tabular}{ll}
Jméno:&Martin Lukeš\\
Cvičící:&Pavel Pták 
\end{tabular}
\end{flushright}
\section*{\begin{center}Úloha 209.\end{center}}
\begin{itemize}
\renewcommand{\labelitemi}{$\filledtriangleright$}
\item \textbf{Příklad 209.1}

Nalezněte nejmenší nezáporné řešení soustavy:
\begin{eqnarray*}
x\equiv&57\,\bmod\,61\\
x\equiv&2\,\bmod\,49\\
x\equiv&39\,\bmod\,41\\
x\equiv&8\,\bmod\,39\\
x\equiv&24\,\bmod\,38\\
\end{eqnarray*}
Nalezené řešení uvažujte jako 10-ti místné a výsledek rozdělte na posloupnost dvouciferných čísel. Tato čísla označte po řadě \(c_0,\ldots,c_4\).
\item \textbf{Řešení 209.1}

Řešení příkladu je posloupnost \(c_0=8,c_1=33,c_2=3,c_3=28,c_4=32\)
\item \textbf{Příklad 209.2}

Pracujte v \( \textbf{Z}_{38}\) a dešifrujte zprávu
\[(4,3,9,5,5)\]
víte-li, že byla použita šifra s šifrovacím klíčem
$$\left(\begin{array}{ccccc}
23&37&c_0&20&37\\c_1&3&22&14&24\\30&22&16&37&c_2\\23&c_3&4&3&15\\5&20&35&c_4&19
\end{array}\right)$$
kde \(c_0,\ldots,c_4\) jsou hodnoty získané z předešlé úlohy. Dešifrovanou zprávu převeďte na text za použití kódování:
$$\begin{array}{ccc}
A\mapsto 01 & B\mapsto 02 & C\mapsto 03 \\
&\vdots&\\
\cdots & Z\mapsto 26 &
\end{array}
$$
Dále dešifrujte zprávu
\[(10,3,1,20,5)\]
Zpráva byla zašifrována stejnou metodou jako v předchozím případě. Výslednou zprávu nepřevádějte na text, ale označte její hodnoty po řadě \(r_0,\ldots,r_4.\) Tyto hodnoty použijete v následující úloze.
\item \textbf{Řešení 209.2}

Dešifrovaná zpráva je \textbf{JITRO} a hledaná posloupnost je \(r_0=9,r_1=6,r_2=5,r_3=2,r_4=5\)
\item \textbf{Příklad 209.3}

Dešifrujte zprávu
$$\left(\begin{array}{lr}
\multicolumn{2}{l}{59895979813927177746967515458370703674904862688299943711385748532}\\
\multicolumn{2}{l}{38579287523012364759505752650789801965240048793058502452323154189}\\
\multicolumn{2}{l}{73577778512769049562550594164399695508441243715632879037873903869}\\
40659&\bullet\\
\end{array}\right)$$
víte-li, že byla použita RSA šifra s šifrovacím klíčem
$$(n,e) = \left(\begin{array}{lr}
\multicolumn{2}{l}{846035240390623252{r_0}242597569586650483985160468942143109{r_1}710313678}\\
\multicolumn{2}{l}{0423971170401758927{r_2}178644913706996503276072866527941075206523174}\\
\multicolumn{2}{l}{124845017202947767074944{r_3}963507478095108{r_4}454295643692848636280219}\\
55089&\bullet\\
\multicolumn{2}{l}{16327919883770007484751746844379775323670973070162104248460495280}\\
41115257776462404103021141134828851&\bullet\\
\end{array}\right)$$
kde  \(r_0,\ldots,r_4.\) jsou hodnoty získané z předešlé úlohy. Dešifrovanou zprávu převeďte na text. Bylo použito ASCII kódování.
\item \textbf{Řešení 209.3}

Dešifrovaná zpráva je \textbf{miss Dubedat? Yes, do bedad. And she did bedad. Huguenot name I expect} .
\end{itemize}
\newpage\subsection*{\begin{center}Podrobné řešení\end{center}}
\begin{itemize}
\renewcommand{\labelitemi}{$\filledtriangleright$}
\item \textbf{Příklad 209.1}

Řešení bude v \(\textbf{Z}_M\), kde \(M = 61\cdot49\cdot41\cdot39\cdot38 = 181617618\)
$$x = 57\cdot(49\cdot 41\cdot 39\cdot 38\cdot a_0) + 2\cdot(61\cdot 41\cdot 39\cdot 38\cdot a_1) + 39\cdot(61\cdot 49\cdot 39\cdot 38\cdot a_2) + $$
$$ + 8\cdot(61\cdot 49\cdot 41\cdot 38\cdot a_3) + 24\cdot(61\cdot 49\cdot 41\cdot 39\cdot a_4)$$
$$a_0 = (49\cdot 41\cdot 39\cdot 38)^{-1} \bmod 61 = (50)^{-1} \bmod 61 = 11$$
\begin{center}Výpočet \(50^{-1} \bmod 61\) rozšířeným euklidovým algoritmem

\begin{tabular}{cc|cc}&61&1&0\\
&50&0&1\\
\hline
1&11&1&-1\\
4&6&-4&5\\
1&5&5&-6\\
1&1&-9&11\\
5&0
\end{tabular}

\(1 = 11\cdot 50 + -9\cdot 61\)

\(50^{-1} \bmod 61 = 11\)
\end{center}
$$a_1 = (61\cdot 41\cdot 39\cdot 38)^{-1} \bmod 49 = (24)^{-1} \bmod 49 = 47$$
\begin{center}Výpočet \(24^{-1} \bmod 49\) rozšířeným euklidovým algoritmem

\begin{tabular}{cc|cc}&49&1&0\\
&24&0&1\\
\hline
2&1&1&-2\\
24&0
\end{tabular}

\(1 = -2\cdot 24 + 1\cdot 49\)

\(24^{-1} \bmod 49 = -2 = 47\)
\end{center}
$$a_2 = (61\cdot 49\cdot 39\cdot 38)^{-1} \bmod 41 = (17)^{-1} \bmod 41 = 29$$
\begin{center}Výpočet \(17^{-1} \bmod 41\) rozšířeným euklidovým algoritmem

\begin{tabular}{cc|cc}&41&1&0\\
&17&0&1\\
\hline
2&7&1&-2\\
2&3&-2&5\\
2&1&5&-12\\
3&0
\end{tabular}

\(1 = -12\cdot 17 + 5\cdot 41\)

\(17^{-1} \bmod 41 = -12 = 29\)
\end{center}
$$a_3 = (61\cdot 49\cdot 41\cdot 38)^{-1} \bmod 39 = (28)^{-1} \bmod 39 = 7$$
\begin{center}Výpočet \(28^{-1} \bmod 39\) rozšířeným euklidovým algoritmem

\begin{tabular}{cc|cc}&39&1&0\\
&28&0&1\\
\hline
1&11&1&-1\\
2&6&-2&3\\
1&5&3&-4\\
1&1&-5&7\\
5&0
\end{tabular}

\(1 = 7\cdot 28 + -5\cdot 39\)

\(28^{-1} \bmod 39 = 7\)
\end{center}
$$a_4 = (61\cdot 49\cdot 41\cdot 39)^{-1} \bmod 38 = (37)^{-1} \bmod 38 = 37$$
\begin{center}Výpočet \(37^{-1} \bmod 38\) rozšířeným euklidovým algoritmem

\begin{tabular}{cc|cc}&38&1&0\\
&37&0&1\\
\hline
1&1&1&-1\\
37&0
\end{tabular}

\(1 = -1\cdot 37 + 1\cdot 38\)

\(37^{-1} \bmod 38 = -1 = 37\)
\end{center}
$$x = 57\cdot(49\cdot 41\cdot 39\cdot 38\cdot 11) + 2\cdot(61\cdot 41\cdot 39\cdot 38\cdot 47) + 39\cdot(61\cdot 49\cdot 39\cdot 38\cdot 29) + $$
$$ + 8\cdot(61\cdot 49\cdot 41\cdot 38\cdot 7) + 24\cdot(61\cdot 49\cdot 41\cdot 39\cdot 37)$$
$$x = 11730089912 = 106562360 \bmod 181617618$$
Hledá se 5. nejmenší řešení
$$x_5 = 106562360 + 4\cdot 181617618 = 833032832$$
Řešení příkladu je posloupnost \(c_0=8,c_1=33,c_2=3,c_3=28,c_4=32\)
\item \textbf{Příklad 209.2}

Šifrovací klíč:
$$\left(\begin{array}{ccccc}
23&37&8&20&37\\
33&3&22&14&24\\
30&22&16&37&3\\
23&28&4&3&15\\
5&20&35&32&19\\
\end{array}\right)$$
Rozšířím matici o jednotkovou směrem vpravo
$$\left(\begin{array}{cccccccccc}
23&37&8&20&37&1&0&0&0&0\\
33&3&22&14&24&0&1&0&0&0\\
30&22&16&37&3&0&0&1&0&0\\
23&28&4&3&15&0&0&0&1&0\\
5&20&35&32&19&0&0&0&0&1\\
\end{array}\right)$$
Upravuji na horní trojúhelníkovou matici.
Násobím řádek 1 číslem 5
$$\left(\begin{array}{cccccccccc}
1&33&2&24&33&5&0&0&0&0\\
33&3&22&14&24&0&1&0&0&0\\
30&22&16&37&3&0&0&1&0&0\\
23&28&4&3&15&0&0&0&1&0\\
5&20&35&32&19&0&0&0&0&1\\
\end{array}\right)$$
Odečítám \(33\times\) řádek 1 od řádku 2
$$\left(\begin{array}{cccccccccc}
1&33&2&24&33&5&0&0&0&0\\
0&16&32&20&37&25&1&0&0&0\\
30&22&16&37&3&0&0&1&0&0\\
23&28&4&3&15&0&0&0&1&0\\
5&20&35&32&19&0&0&0&0&1\\
\end{array}\right)$$
Odečítám \(30\times\) řádek 1 od řádku 3
$$\left(\begin{array}{cccccccccc}
1&33&2&24&33&5&0&0&0&0\\
0&16&32&20&37&25&1&0&0&0\\
0&20&32&1&1&2&0&1&0&0\\
23&28&4&3&15&0&0&0&1&0\\
5&20&35&32&19&0&0&0&0&1\\
\end{array}\right)$$
Odečítám \(23\times\) řádek 1 od řádku 4
$$\left(\begin{array}{cccccccccc}
1&33&2&24&33&5&0&0&0&0\\
0&16&32&20&37&25&1&0&0&0\\
0&20&32&1&1&2&0&1&0&0\\
0&29&34&21&16&37&0&0&1&0\\
5&20&35&32&19&0&0&0&0&1\\
\end{array}\right)$$
Odečítám \(5\times\) řádek 1 od řádku 5
$$\left(\begin{array}{cccccccccc}
1&33&2&24&33&5&0&0&0&0\\
0&16&32&20&37&25&1&0&0&0\\
0&20&32&1&1&2&0&1&0&0\\
0&29&34&21&16&37&0&0&1&0\\
0&7&25&26&6&13&0&0&0&1\\
\end{array}\right)$$
Přičítám řádek 4 k řádku 2
$$\left(\begin{array}{cccccccccc}
1&33&2&24&33&5&0&0&0&0\\
0&7&28&3&15&24&1&0&1&0\\
0&20&32&1&1&2&0&1&0&0\\
0&29&34&21&16&37&0&0&1&0\\
0&7&25&26&6&13&0&0&0&1\\
\end{array}\right)$$
Násobím řádek 2 číslem 11
$$\left(\begin{array}{cccccccccc}
1&33&2&24&33&5&0&0&0&0\\
0&1&4&33&13&36&11&0&11&0\\
0&20&32&1&1&2&0&1&0&0\\
0&29&34&21&16&37&0&0&1&0\\
0&7&25&26&6&13&0&0&0&1\\
\end{array}\right)$$
Odečítám \(20\times\) řádek 2 od řádku 3
$$\left(\begin{array}{cccccccccc}
1&33&2&24&33&5&0&0&0&0\\
0&1&4&33&13&36&11&0&11&0\\
0&0&28&25&7&4&8&1&8&0\\
0&29&34&21&16&37&0&0&1&0\\
0&7&25&26&6&13&0&0&0&1\\
\end{array}\right)$$
Odečítám \(29\times\) řádek 2 od řádku 4
$$\left(\begin{array}{cccccccccc}
1&33&2&24&33&5&0&0&0&0\\
0&1&4&33&13&36&11&0&11&0\\
0&0&28&25&7&4&8&1&8&0\\
0&0&32&14&19&19&23&0&24&0\\
0&7&25&26&6&13&0&0&0&1\\
\end{array}\right)$$
Odečítám \(7\times\) řádek 2 od řádku 5
$$\left(\begin{array}{cccccccccc}
1&33&2&24&33&5&0&0&0&0\\
0&1&4&33&13&36&11&0&11&0\\
0&0&28&25&7&4&8&1&8&0\\
0&0&32&14&19&19&23&0&24&0\\
0&0&35&23&29&27&37&0&37&1\\
\end{array}\right)$$
Přičítám řádek 5 k řádku 3
$$\left(\begin{array}{cccccccccc}
1&33&2&24&33&5&0&0&0&0\\
0&1&4&33&13&36&11&0&11&0\\
0&0&25&10&36&31&7&1&7&1\\
0&0&32&14&19&19&23&0&24&0\\
0&0&35&23&29&27&37&0&37&1\\
\end{array}\right)$$
Násobím řádek 3 číslem 35
$$\left(\begin{array}{cccccccccc}
1&33&2&24&33&5&0&0&0&0\\
0&1&4&33&13&36&11&0&11&0\\
0&0&1&8&6&21&17&35&17&35\\
0&0&32&14&19&19&23&0&24&0\\
0&0&35&23&29&27&37&0&37&1\\
\end{array}\right)$$
Odečítám \(32\times\) řádek 3 od řádku 4
$$\left(\begin{array}{cccccccccc}
1&33&2&24&33&5&0&0&0&0\\
0&1&4&33&13&36&11&0&11&0\\
0&0&1&8&6&21&17&35&17&35\\
0&0&0&24&17&31&11&20&12&20\\
0&0&35&23&29&27&37&0&37&1\\
\end{array}\right)$$
Odečítám \(35\times\) řádek 3 od řádku 5
$$\left(\begin{array}{cccccccccc}
1&33&2&24&33&5&0&0&0&0\\
0&1&4&33&13&36&11&0&11&0\\
0&0&1&8&6&21&17&35&17&35\\
0&0&0&24&17&31&11&20&12&20\\
0&0&0&9&9&14&12&29&12&30\\
\end{array}\right)$$
Přičítám řádek 5 k řádku 4
$$\left(\begin{array}{cccccccccc}
1&33&2&24&33&5&0&0&0&0\\
0&1&4&33&13&36&11&0&11&0\\
0&0&1&8&6&21&17&35&17&35\\
0&0&0&33&26&7&23&11&24&12\\
0&0&0&9&9&14&12&29&12&30\\
\end{array}\right)$$
Násobím řádek 4 číslem 15
$$\left(\begin{array}{cccccccccc}
1&33&2&24&33&5&0&0&0&0\\
0&1&4&33&13&36&11&0&11&0\\
0&0&1&8&6&21&17&35&17&35\\
0&0&0&1&10&29&3&13&18&28\\
0&0&0&9&9&14&12&29&12&30\\
\end{array}\right)$$
Odečítám \(9\times\) řádek 4 od řádku 5
$$\left(\begin{array}{cccccccccc}
1&33&2&24&33&5&0&0&0&0\\
0&1&4&33&13&36&11&0&11&0\\
0&0&1&8&6&21&17&35&17&35\\
0&0&0&1&10&29&3&13&18&28\\
0&0&0&0&33&19&23&26&2&6\\
\end{array}\right)$$
Násobím řádek 5 číslem 15
$$\left(\begin{array}{cccccccccc}
1&33&2&24&33&5&0&0&0&0\\
0&1&4&33&13&36&11&0&11&0\\
0&0&1&8&6&21&17&35&17&35\\
0&0&0&1&10&29&3&13&18&28\\
0&0&0&0&1&19&3&10&30&14\\
\end{array}\right)$$
Provádím úpravu na jednotkovou matici.
Odečítám \(10\times\) řádek 5 od řádku 4
$$\left(\begin{array}{cccccccccc}
1&33&2&24&33&5&0&0&0&0\\
0&1&4&33&13&36&11&0&11&0\\
0&0&1&8&6&21&17&35&17&35\\
0&0&0&1&0&29&11&27&22&2\\
0&0&0&0&1&19&3&10&30&14\\
\end{array}\right)$$
Odečítám \(6\times\) řádek 5 od řádku 3
$$\left(\begin{array}{cccccccccc}
1&33&2&24&33&5&0&0&0&0\\
0&1&4&33&13&36&11&0&11&0\\
0&0&1&8&0&21&37&13&27&27\\
0&0&0&1&0&29&11&27&22&2\\
0&0&0&0&1&19&3&10&30&14\\
\end{array}\right)$$
Odečítám \(13\times\) řádek 5 od řádku 2
$$\left(\begin{array}{cccccccccc}
1&33&2&24&33&5&0&0&0&0\\
0&1&4&33&0&17&10&22&1&8\\
0&0&1&8&0&21&37&13&27&27\\
0&0&0&1&0&29&11&27&22&2\\
0&0&0&0&1&19&3&10&30&14\\
\end{array}\right)$$
Odečítám \(33\times\) řádek 5 od řádku 1
$$\left(\begin{array}{cccccccccc}
1&33&2&24&0&24&15&12&36&32\\
0&1&4&33&0&17&10&22&1&8\\
0&0&1&8&0&21&37&13&27&27\\
0&0&0&1&0&29&11&27&22&2\\
0&0&0&0&1&19&3&10&30&14\\
\end{array}\right)$$
Odečítám \(8\times\) řádek 4 od řádku 3
$$\left(\begin{array}{cccccccccc}
1&33&2&24&0&24&15&12&36&32\\
0&1&4&33&0&17&10&22&1&8\\
0&0&1&0&0&17&25&25&3&11\\
0&0&0&1&0&29&11&27&22&2\\
0&0&0&0&1&19&3&10&30&14\\
\end{array}\right)$$
Odečítám \(33\times\) řádek 4 od řádku 2
$$\left(\begin{array}{cccccccccc}
1&33&2&24&0&24&15&12&36&32\\
0&1&4&0&0&10&27&5&35&18\\
0&0&1&0&0&17&25&25&3&11\\
0&0&0&1&0&29&11&27&22&2\\
0&0&0&0&1&19&3&10&30&14\\
\end{array}\right)$$
Odečítám \(24\times\) řádek 4 od řádku 1
$$\left(\begin{array}{cccccccccc}
1&33&2&0&0&12&17&10&2&22\\
0&1&4&0&0&10&27&5&35&18\\
0&0&1&0&0&17&25&25&3&11\\
0&0&0&1&0&29&11&27&22&2\\
0&0&0&0&1&19&3&10&30&14\\
\end{array}\right)$$
Odečítám \(4\times\) řádek 3 od řádku 2
$$\left(\begin{array}{cccccccccc}
1&33&2&0&0&12&17&10&2&22\\
0&1&0&0&0&18&3&19&23&12\\
0&0&1&0&0&17&25&25&3&11\\
0&0&0&1&0&29&11&27&22&2\\
0&0&0&0&1&19&3&10&30&14\\
\end{array}\right)$$
Odečítám \(2\times\) řádek 3 od řádku 1
$$\left(\begin{array}{cccccccccc}
1&33&0&0&0&16&5&36&34&0\\
0&1&0&0&0&18&3&19&23&12\\
0&0&1&0&0&17&25&25&3&11\\
0&0&0&1&0&29&11&27&22&2\\
0&0&0&0&1&19&3&10&30&14\\
\end{array}\right)$$
Odečítám \(33\times\) řádek 2 od řádku 1
$$\left(\begin{array}{cccccccccc}
1&0&0&0&0&30&20&17&35&22\\
0&1&0&0&0&18&3&19&23&12\\
0&0&1&0&0&17&25&25&3&11\\
0&0&0&1&0&29&11&27&22&2\\
0&0&0&0&1&19&3&10&30&14\\
\end{array}\right)$$
Inverzní matice je pravá část vedle jednotkové matice
$$\left(\begin{array}{ccccc}
30&20&17&35&22\\
18&3&19&23&12\\
17&25&25&3&11\\
29&11&27&22&2\\
19&3&10&30&14\\
\end{array}\right)$$
Výsledná inverzní matice je klíčem k dešifrování.
Vynásobíme-li tuto matici zašifrovanou zprávou, získáme dešifrovanou zprávu.
Násobíme zprávou:
$$\left(\begin{array}{c}
4\\
3\\
9\\
5\\
5\\
\end{array}\right)$$
Po vynásobení:
$$\left(\begin{array}{c}
10\\
9\\
20\\
18\\
15\\
\end{array}\right)$$
Po převedení na písmena získáme slovo 
JITRO.
Nyní stejným způsobem dešifrujeme druhou zprávu.
Dešifrovacím klíčem násobíme zprávu
$$\left(\begin{array}{c}
10\\
3\\
1\\
20\\
5\\
\end{array}\right)$$
Po vynásobení:
$$\left(\begin{array}{c}
9\\
6\\
5\\
2\\
5\\
\end{array}\right)$$
Hledaná posloupnost je \(r_0=9,r_1=6,r_2=5,r_3=2,r_4=5\)
\item \textbf{Příklad 209.3}
$$t = \lceil \sqrt{n} \rceil = \left(\begin{array}{lr}\multicolumn{2}{l}{91980173971928496941}\\
\multicolumn{2}{l}{38754448114526310137}\\
\multicolumn{2}{l}{19647897400304541041}\\
\multicolumn{2}{l}{96861407968222207441}\\
56921994237453980183&\bullet\\
\end{array}\right)$$
$$x = t^2 - n $$
$$x = 9163237527176357006569142903120797645120534758400$$
$$s = \sqrt{x} = 3027083997377072621653280$$
\(x\) je přesně druhá mocnina čísla \(s\)
$$a = t + s = \left(\begin{array}{lr}\multicolumn{2}{l}{91980173971928496941}\\
\multicolumn{2}{l}{38754448114526310137}\\
\multicolumn{2}{l}{19647897400304541041}\\
\multicolumn{2}{l}{96861407968222207441}\\
56921994237453980183&\bullet\\
\end{array}\right) + \left(\begin{array}{lr}\multicolumn{2}{l}{30270839973770726216}\\
53280&\bullet\\
\end{array}\right)$$
$$a = \left(\begin{array}{lr}\multicolumn{2}{l}{91980173971928496941}\\
\multicolumn{2}{l}{38754448114526310137}\\
\multicolumn{2}{l}{19647897400304541041}\\
\multicolumn{2}{l}{96861407968222237712}\\
40919371310075633463&\bullet\\
\end{array}\right)$$
$$b = t - s = \left(\begin{array}{lr}\multicolumn{2}{l}{91980173971928496941}\\
\multicolumn{2}{l}{38754448114526310137}\\
\multicolumn{2}{l}{19647897400304541041}\\
\multicolumn{2}{l}{96861407968222207441}\\
56921994237453980183&\bullet\\
\end{array}\right) - \left(\begin{array}{lr}\multicolumn{2}{l}{30270839973770726216}\\
53280&\bullet\\
\end{array}\right)$$
$$b = \left(\begin{array}{lr}\multicolumn{2}{l}{91980173971928496941}\\
\multicolumn{2}{l}{38754448114526310137}\\
\multicolumn{2}{l}{19647897400304541041}\\
\multicolumn{2}{l}{96861407968222177170}\\
72924617164832326903&\bullet\\
\end{array}\right)$$
$$\phi = (a-1)\cdot(b-1) = \left(\begin{array}{lr}\multicolumn{2}{l}{91980173971928496941}\\
\multicolumn{2}{l}{38754448114526310137}\\
\multicolumn{2}{l}{19647897400304541041}\\
\multicolumn{2}{l}{96861407968222237712}\\
40919371310075633462&\bullet\\
\end{array}\right)\cdot \left(\begin{array}{lr}\multicolumn{2}{l}{91980173971928496941}\\
\multicolumn{2}{l}{38754448114526310137}\\
\multicolumn{2}{l}{19647897400304541041}\\
\multicolumn{2}{l}{96861407968222177170}\\
72924617164832326902&\bullet\\
\end{array}\right)$$
$$\phi = \left(\begin{array}{lr}\multicolumn{2}{l}{84603524039062325292}\\
\multicolumn{2}{l}{42597569586650483985}\\
\multicolumn{2}{l}{16046894214310967103}\\
\multicolumn{2}{l}{13678042397117040175}\\
\multicolumn{2}{l}{89275178644913706994}\\
\multicolumn{2}{l}{66367259342795800224}\\
\multicolumn{2}{l}{74556335512195829897}\\
\multicolumn{2}{l}{63651972274335214266}\\
\multicolumn{2}{l}{81058135149009880760}\\
55440875153113994724&\bullet\\
\end{array}\right)$$
$$D = E^{-1} \bmod \phi$$
$$D = \left(\begin{array}{lr}\multicolumn{2}{l}{16327919883770007484}\\
\multicolumn{2}{l}{75174684437977532367}\\
\multicolumn{2}{l}{09730701621042484604}\\
\multicolumn{2}{l}{95280411152577764624}\\
04103021141134828851&\bullet\\
\end{array}\right)^{-1} mod \left(\begin{array}{lr}\multicolumn{2}{l}{84603524039062325292}\\
\multicolumn{2}{l}{42597569586650483985}\\
\multicolumn{2}{l}{16046894214310967103}\\
\multicolumn{2}{l}{13678042397117040175}\\
\multicolumn{2}{l}{89275178644913706994}\\
\multicolumn{2}{l}{66367259342795800224}\\
\multicolumn{2}{l}{74556335512195829897}\\
\multicolumn{2}{l}{63651972274335214266}\\
\multicolumn{2}{l}{81058135149009880760}\\
55440875153113994724&\bullet\\
\end{array}\right)$$
$$D = \left(\begin{array}{lr}\multicolumn{2}{l}{43038607273697049376}\\
\multicolumn{2}{l}{64262499637183147316}\\
\multicolumn{2}{l}{80197417838197702440}\\
\multicolumn{2}{l}{36173382573276173422}\\
\multicolumn{2}{l}{75849006698725766510}\\
\multicolumn{2}{l}{40122563151868500163}\\
\multicolumn{2}{l}{78291973902510792663}\\
\multicolumn{2}{l}{37231900478820784596}\\
\multicolumn{2}{l}{84638604624337099871}\\
80165629531505368355&\bullet\\
\end{array}\right)$$
$$OT = ST^{D} \bmod n$$
$$OT = \left(\begin{array}{lr}\multicolumn{2}{l}{59895979813927177746}\\
\multicolumn{2}{l}{96751545837070367490}\\
\multicolumn{2}{l}{48626882999437113857}\\
\multicolumn{2}{l}{48532385792875230123}\\
\multicolumn{2}{l}{64759505752650789801}\\
\multicolumn{2}{l}{96524004879305850245}\\
\multicolumn{2}{l}{23231541897357777851}\\
\multicolumn{2}{l}{27690495625505941643}\\
\multicolumn{2}{l}{99695508441243715632}\\
87903787390386940659&\bullet\\
\end{array}\right)^{\left(\begin{array}{lr}\multicolumn{2}{l}{43038607273697049376}\\
\multicolumn{2}{l}{64262499637183147316}\\
\multicolumn{2}{l}{80197417838197702440}\\
\multicolumn{2}{l}{36173382573276173422}\\
\multicolumn{2}{l}{75849006698725766510}\\
\multicolumn{2}{l}{40122563151868500163}\\
\multicolumn{2}{l}{78291973902510792663}\\
\multicolumn{2}{l}{37231900478820784596}\\
\multicolumn{2}{l}{84638604624337099871}\\
80165629531505368355&\bullet\\
\end{array}\right)} \bmod n$$
$$OT = \left(\begin{array}{lr}\multicolumn{2}{l}{17158005656493515016}\\
\multicolumn{2}{l}{10392672214768226748}\\
\multicolumn{2}{l}{81544208109242696727}\\
\multicolumn{2}{l}{93359731828861842284}\\
\multicolumn{2}{l}{93852482531095526041}\\
\multicolumn{2}{l}{34945886350970090225}\\
\multicolumn{2}{l}{83409274146899504331}\\
\multicolumn{2}{l}{49941147150059439527}\\
528982893&\bullet\\
\end{array}\right)$$
Převedením na ASCII znaky získáme:\textbf{miss Dubedat? Yes, do bedad. And she did bedad. Huguenot name I expect}
\end{itemize}
\end{document}
